%sri
\documentclass[conference]{IEEEtran}
\IEEEoverridecommandlockouts
% The preceding line is only needed to identify funding in the first footnote. If that is unneeded, please comment it out.
\usepackage{cite}
\usepackage{amsmath,amssymb,amsfonts}
\usepackage{algorithmic}
\usepackage{graphicx}
\usepackage{textcomp}
\def\BibTeX{{\rm B\kern-.05em{\sc i\kern-.025em b}\kern-.08em
    T\kern-.1667em\lower.7ex\hbox{E}\kern-.125emX}}
\begin{document}

\title{As Much Resilience As You Want: \\Application-tailored Resilience in Legion}

%\author{\IEEEauthorblockN{Karthik Murthy} \and
%\IEEEauthorblockN{Mike Bauer} \and 
%\IEEEauthorblockN{Manolis Papadakis} \and
%\IEEEauthorblockN{Kyushick Lee} \and
%\IEEEauthorblockN{Yongkee Kwon} \and
%\IEEEauthorblockN{Wonchan Lee} \and
%\IEEEauthorblockN{Todd Warszawski} \and 
%\IEEEauthorblockN{Elliott Slaughter} \and
%\IEEEauthorblockN{Sean Treichler} \and
%\IEEEauthorblockN{Alex Aiken} \and
%\IEEEauthorblockN{Mattan Erez}
%}

\maketitle

\begin{abstract}
Resilient is an important aspect of scaling applications on
large clusters. As we approach parallelism profiles of several millions and long running applications, we need to ensure that ineffect ``we reach the end".
Defining the policy interaction with the programming model features necessitates that we revisit the memory concistency model. 
Recoverability, a critical step in
resilience, opens the door to optimizations such as speculation. We also
evaluate this.  \dots
\end{abstract}

\begin{IEEEkeywords}
resilience, legion
\end{IEEEkeywords}

%sri
\section{Introduction}

A failure during a large-scale execution of any application on an extreme-scale
system leads to loss of time and money, and can cause a nightmare to any
application developer. While failures could be due to application bugs,
failures due to errors not detectable by error correcting code (ECC) are on the
rise.  For example, Geist states that such failures are a common occurrence on
Oak Ridge National Laboratories leadership class machines~\cite{errors_ecc},
and Schroeder and Gibson state that a large number of CPU and memory failures
were from parity errors after tracking a five-year log of hardware replacements
for a $765$ node high-performance computing cluster~\cite{schroeder_gibson}.
Echoing the sentiment of Snir et al.~\cite{snir}, we believe it is critical to
overcome the effect of such failures through a productive and performant resilience 
interface in the parallel programming model so that a developer can make avail 
the performance on extreme- and upcoming exa-scale parallel systems.

The traditional solutions to address failures during large-scale parallel
executions includes the application developer: (a) manually takes periodic
checkpoints, and manually controls the rollback and restart from specific
checkpoints~\cite{checkpointrestart}, (b) employs a semi-automatic approach, 
where the developer employs programming constructs that have well defined 
semantics in the event of failures, e.g, \texttt{finish} blocks in X10~\cite{X10}.
Here, the parallel programming model runtime automatically corrects itself to 
handle further execution in the event of a failure but limited to the beginning
of these specific programming constructs. (c) employs a sophistica

In x10, use programming language constructs such as resilient finish blocks to
capture their completing along with useful semnatic guarantees. 

Autocorrecting, where the programmer labels computations which form. 

We believe these solutions are good, but not flexible to demand the
complexities of current extreme-scale and upcoming exa-scale systems. Its is
not at one point, every mapping needs to make a decision, support for checking
whther there is a tradeoff between recomputing the dag vs .. this has to be
dynamic. We also believe that these decision should be in a deferred execution
state, not the whole computation stopping.


To drive this point, consider the figure . A simple computation task graph, X10
would allow this, PARSEC would allow this, Charm++ would focus on the stores of
alpha, x and y. However, we believe that this solution.  soemtimes you have a
resilient store where you map alpha, x but sometimes not. Think here.

The right vehicle to demonstrate such a dynamic decision for checkpoints, along
with a deferred execution model is the Legion programming model. The strengths
of this include a decoupled scheduling, mapping, and execution analysis stages
that drives this solution.

\begin{figure}
\centering
\includegraphics[width=.42\textwidth]{images/spectrum_x10_parsec_legion_policies.png}
\caption{A spectrum of resilience mechanism supported by different
state-of-the-art parallel runtimes along with the proposed resilience strategy
for Legion.} 
\end{figure}

Our contributions are as follows: 
\begin{itemize} 
\item the most flexible resilience mechanism 
\item dynamic decisions for checkpointing 
\item auto rollback and recovery 
\item garbage collector that will also collect previous 
checkpoints based on a novel post-dominator algorithm.  
\item a discussion of the semantics of regions when there is rollback 
\end{itemize}

The rest of the paper is organized as follows:

%With the advent of productive programming models like Legion~\cite{legion},
%X10~\cite{x10}, PARSEC~\cite{parsec}, CHARM++~\cite{charm++}, and others,
%programming extreme-scale systems is not as significant a challenge as
%addressing other 
%
%how is it different from x10, parsec, charm++\\
%	- they also allow tasks to be marked as resilience\\
%	- x10 allows finish blocks, exception semantics.\\
%	- what are the exception semantics that we are providing\\
%
%what about local vs global recovery\\
%	- can we recover from a node failure\\
%	- can we recover from an exception, what are the semantics that we provide
%	  here\\
%	- can we recover from ECC errors, \\
%	- can we fix these errors ?\\
%	- can we recover from I/O errors ?\\ 
%	- can we recover from non-SingleTasks, what is recovery for index space
%	  tasks, must epoch tasks\\
%
%
%experiments: circuit, miniAero, Soleil-X, stencil, (S3D ?)\\
%


%sri

\section{Overview of Resilience in Legion}


%sri

\section{Interaction of Resilience Policy with Legion Features} 

which are the legion features of importance ?

put a figure of interaction.

what is the memory consistency angle here ?


\section{Implementation of Resilience}

\subsection{\texttt{need\_preserve} Implementation}

tagging region instances
tagging tasks

-the mapper marks an instance as persistent, i.e., vector<vector<bool>> persistent
-in finalize\_map\_task, the singleTask sees this and notes it down in the Individual\_task's persistent\_tasks list
-when trigger\_complete is called on the task, 

	1) inside line 5560, invalidate\_region\_tree\_contexts, 
		inside which we have runtime->forest->invalidate\_versions, we do not do on region[idx].region.
		we also do not do the instance\_top\_views
		
	2) we retain the mark on the task as allowed\_for\_gc.

   3) we go to the incoming of this task, and just like verified\_regions[true], we mark outgoing\_edge\_dominated[true]
      if all the outgoing of a task is marked as true, then we change allowed\_for\_gc = true.


Discussion Points for today
--------------------------
1) allow persistence on a subset of mapped instances in map\_task ? Pro:
flexibility, Con: if a checkpoint is to be considered useful for a restart, not
having the full set of inputs checkpointed seems contradictory. 
2) verify\_regions tracks op dependencies after physical dependence analysis, correct ? 
3) a discussion on persistence inside map\_task call 


discussed design:

1) build a function set\_hardened\_instnace(instance,task) along the lines of the set\_gc\_priority(instance, never, task) inside the mapper call
2) inside that mark a task's incoming edges as saying that it leads to a hardened instance(task)
3) the garbage collector will basically collect a task whose outgoing edges are all marked as verified/hardened.
4) if a task is gc'ed, then it marks all its incoming edges as leads to a hardened instance.
5) steps 1-4 will be based on set\_garbage\_collection\_priority and how verified\_regions are set. We will be adding a new list to each task, similar to verified\_regions, that will represent edges\_ending\_in\_hardened\_regions.



\subsubsection{Interaction of need\_preserve with the commit wavefront}
\subsubsection{Obtaining dependence graph in the mapper before calling need\_preserve} 


\subsection{ProfilingResponse callback used}
while using the profiling measurement reporting infrastructure as-is causes overhead, this is because of the invoking of a mapper profiling response function. We do not need that for resilience, so the resilience callback would short circuit it.

task launch -> porfiling reported event is there -> when it is triggered -> \texttt{singletask::profiling\_response} is invoked -> handle things
   ----- in here, there is no need to call the mapper side yet. So, avoid one overhead there, maybe this was never called and so , this is not an optimization. ok, back to square one. 

 


%sri


there are three different wavefronts in Legion, we can speculate on any of them. 


From a speculation perspective, are they different ? 
Are we novel, since we have these three different wavefronts ? 
Can we navigate through this, like the blanks 


1) execution wavefront 
what does it mean to speculate here, does the other steps have to be complete before we do this. 


2) mapping wavefront

3) dependence analysis wavefront


-- see mike's 6 wavefront answer.

There are three different wavefronts, there could

 
mike - speculation is more about tracking the resolution wavefront while resilience is more about tracking the commit wavefront, but the when things go bad, then i think the machinery to restart the mapping and execution wavefronts should be the same


%sri

\section{Experiments}

\subsection{Index Tasks}

\paragraph{Growth of memory as execution proceeds}
\paragraph{Performance overhead as execution proceeds}



\begin{center}
 \begin{tabular}{||c c c c||} 
 \hline
 Number of Iterations (1000 tasks/index launch, seconds) & Time Without Resilience Without lg:resilient & Time Without Resilienct With lg:resilient & Time With Resilience without lg:resilient \\ [0.25ex] 
 \hline\hline
100 &  7.2 & 6.6 & 23.6\\ 
 \hline
200 &  13.8 & 13.3 & 46.3\\ 
 \hline
400 &  26.1 & 26.5 & 94.4\\ 
 \hline
800 &  55 & 53 & 186.8\\ 
 \hline
1000 &  65 & 66 & 239.9\\ [1ex] 
 \hline
\end{tabular}
\end{center}

\begin{figure}
\includegraphics[width=\textwidth]{images/index_tasks_time.png}
\caption{Total time taken by 02\_index\_tasks.}
\end{figure}


\begin{figure}
\includegraphics[width=\textwidth]{images/index_tasks_memory.png}
\caption{Total Memory footprint by 02\_index\_tasks.}
\end{figure}


\begin{center}
 \begin{tabular}{||c c c c||} 
 \hline
 Number of Iterations (1000 tasks/index launch, MB) & Memory\% Without Resilience Without lg:resilient & Memory\% Without Resilienct With lg:resilient & Memory\% With Resilience without lg:resilient \\ [0.25ex] 
 \hline\hline
100 &  18 & 140 & 107 \\ 
 \hline
200 &  22 & 263 & 176 \\ 
 \hline
400 &  24 & 509 & 245 \\ 
 \hline
800 &  26 & 1002 & 428\\ 
 \hline
1000 & 26 & 1248 & 518\\ [1ex] 
 \hline
\end{tabular}
\end{center}




\subsection{Stencil}



\subsection{local recover vs global recovery}

\subsection{compute/comm vs no-failure/single failure/multi-failure}

\subsection{S3D, Pennant, Stencil, Circuit}

\subsection{Some Interesting Task Graphs for Recovery}





\section{Adaptive Resilience}
\section{Resilience Policy Implemented via Mapper: Example 1 Generic}
\section{Resilience Policy Implemented via Mapper: Example 2 UT Austin}


\input{related_work}

\section{Conclusion}

\section*{Acknowledgment}

\begin{thebibliography}{00}
\bibitem{b1} S. Treichler, M. Bauer, and A. Aiken. Language
support for dynamic, hierarchical data partitioning. In
Object Oriented Programming, Systems, Languages,
and Applications (OOPSLA), 2013.
\end{thebibliography}

\end{document}
